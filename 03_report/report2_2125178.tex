% !TEX program = lualatex
\documentclass{article}
\usepackage{fontspec}
\usepackage{luatexja-fontspec}
\usepackage{luatexja-ruby} % ルビを使用する場合に必要
\usepackage{graphicx}
\usepackage{listings}
\usepackage{amsmath}
\usepackage[top=25truemm,bottom=20truemm,left=20truemm,right=20truemm]{geometry}
\usepackage{here} % For figure placement
%\usepackage{xcolor}

% フォントの設定
%\setmainfont{Noto Serif JP}    % 通常の明朝体
%\setsansfont{Noto Sans JP}     % 通常のゴシック体
%\setmonofont{Noto Sans Mono JP} % 等幅フォント
%\setmainjfont{Noto Serif JP}   % 日本語の明朝体
%\setsansjfont{Noto Sans JP}    % 日本語のゴシック体
%\setmonojfont{Noto Sans Mono JP} % 日本語の等幅フォント

% 図の参照パス
\graphicspath{{./02_code/output/}}

% コードのスタイル設定
\lstset{
  basicstyle=\ttfamily\small, % フォントのサイズとスタイル
  %keywordstyle=\color{blue},  % キーワードの色
  %commentstyle=\color{gray},  % コメントの色
  %stringstyle=\color{red},    % 文字列の色
  numbers=left,               % 行番号を左側に表示
  %numberstyle=\tiny\color{gray}, % 行番号のスタイル
  breaklines=true,            % 長い行を折り返す
	frame=single,               % コードブロックに枠をつける
}
\renewcommand{\lstlistingname}{ソースコード}
\begin{document}
\parindent = 0pt

% タイトル
\title{ミクロデータサイエンス\\Problemset2}
\author{2125178\\廣江友哉}
\date{2024/6/19}
\maketitle


% 段落
\section{囚人と主観的確率}

\subsection{3つの扉}

事前確率として, 扉が正解である事象$X$の確率は$P(X) = \cfrac{1}{3}$とする. また, 看守がいずれかの扉を選ぶ事象を$Y$とする. 例えば, 囚人が Aの扉を選び看守が不正解の扉としてCの扉を開けたとすると, A と B それぞれの扉が正解の扉である事後確率はベイズの定理を用いて以下のように表せる.

\begin{center}
	\begin{gather}
		P(X = A | Y = C) = \cfrac{P(Y = C | X = A)}{P(Y = C)} P(X = A) \\
		P(X = B | Y = C) = \cfrac{P(Y = C | X = B)}{P(Y = C)} P(X = B)
	\end{gather}
\end{center}

ここで, Aの扉が正解だったとすると以下のように具体的な確率を求めることができる.

\begin{center}
	\begin{gather}
		P(Y = C) = \frac{1}{2}\quad P(Y = C | X = A) = \frac{1}{2} \\
		P(Y = C) = \frac{1}{2}\quad P(Y = C | X = B) = 1
	\end{gather}
\end{center}

上記の式は, 囚人が正解の扉を選んでいる場合看守は残りのどちらの扉も選ぶことができるのでその確率を仮に$\cfrac{1}{2}$とし, 囚人が不正解の扉を選んでいる場合に看守は扉を一つしか選択する余地がないので確率が$1$となることを表している.上記の確率の値を使用して, ベイズの定理の式1と式2を計算すると,

\begin{center}
	\begin{gather}
		P(X = A | Y = C) = \cfrac{\frac{1}{2}}{\frac{1}{2}} \times \cfrac{1}{3} = \cfrac{1}{3} \\
		P(X = B | Y = C) = \cfrac{1}{\frac{1}{2}} \times \cfrac{1}{3} = \cfrac{2}{3}
	\end{gather}
\end{center}

したがって, 囚人は A から B へ扉を変更した方が正解の確率が上がるので, 囚人の扉を選び直すという行為は正しいことがわかる.

\subsection{3人の囚人}

囚人が釈放される事象$X$の確率が均等に$\cfrac{1}{3}$とすると, 看守が教える囚人の名前を事象$Y$とする. ここで, 便宜上看守が伝えた処刑される囚人の名前をCとすると, 前問のベイズの定理の式と同じ形になる.

\begin{center}
	\begin{gather}
		P(X = A | Y = C) = \cfrac{P(Y = C | X = A)}{P(Y = C)} P(X = A) \\
		P(X = B | Y = C) = \cfrac{P(Y = C | X = B)}{P(Y = C)} P(X = B)
	\end{gather}
\end{center}

具体的な確率を計算すると,

\begin{center}
	\begin{gather}
		P(Y = C) = \frac{1}{2}\quad P(Y = C | X = A) = \frac{1}{2} \\
		P(Y = C) = \frac{1}{2}\quad P(Y = C | X = B) = 1
	\end{gather}
\end{center}

これを元にベイズの定理に沿って計算を進めると,

\begin{center}
	\begin{gather}
		P(X = A | Y = C) = \cfrac{\frac{1}{2}}{\frac{1}{2}} \times \cfrac{1}{3} = \cfrac{1}{3} \\
		P(X = B | Y = C) = \cfrac{1}{\frac{1}{2}} \times \cfrac{1}{3} = \cfrac{2}{3}
	\end{gather}
\end{center}

したがって, 囚人Aが釈放される事後確率は$\cfrac{1}{3}$となるため問題文の囚人Aは釈放される確率が高まると喜んでいるのは誤りである.

\subsection{比較}

問題AとBのどちらも, 対象となる人物の事前確率と事後確率に変化がないことがわかる.

\section{パレート分布と最尤法}

\subsection{パレート分布の尤度関数}

\begin{center}
	\begin{gather}
		F(x) = \begin{cases} 
			1 - \left( \frac{\beta}{x} \right)^\alpha & \text{if } x \geq \beta, \\
			0 & \text{if } x < \beta 
		\end{cases}
	\end{gather}
\end{center}

密度関数を求めると,

\begin{center}
	\begin{gather}
		f(x) = \begin{cases} 
			\alpha\beta^\alpha x^{-\alpha-1} & \text{if } x \geq \beta \\
			0 & \text{if } x < \beta 
		\end{cases}
	\end{gather}
\end{center}

尤度関数は,

\begin{center}
	\begin{gather}
		L(\alpha, \beta) = \prod_{x = \beta}^{n} \alpha\beta^\alpha x^{-\alpha-1}
	\end{gather}
\end{center}

対数尤度関数は,

\begin{center}
	\begin{math}
    \begin{aligned}
      l(\alpha, \beta) &= \sum_{x = i}^{n} \log\alpha\beta^\alpha x^{-\alpha-1} \\
    &= \sum_{x = i}^{n} (\log\alpha + \alpha\log\beta - (\alpha + 1) \log x) \\
      &= n\log\alpha + n\alpha\log\beta - (\alpha + 1) \sum_{x=i}^{n}\log x
    \end{aligned}
	\end{math}
\end{center}

$\beta$は,$x \geq \beta$より,観測値の最小値が最大値となる.$\alpha$について偏微分すると,
\begin{center}

  \begin{math}
    \begin{aligned}
      \dfrac{\partial l(\alpha, \beta)}{\partial \alpha} &=  \dfrac{n}{\alpha} + n\log\beta - \sum_{x = i}^{n} \log x = 0\\
      \alpha &= \dfrac{n}{n\log\beta - \sum_{x = i}^{n} \log x}
    \end{aligned}
  \end{math}
\end{center}

\subsection{解析解を用いた推定}

$\alpha$の値は$-1.303965$,$\beta$の値は$139128$となった.以下にソースコード\ref{b}を添付する.

\begin{lstlisting}[caption=解析解を用いた推定,label=b]
  # (b)
  sample_size <- 200 
  beta <- min(df$population) 
  alpha_hat <- length(df$population) /
    (length(df$population) * log(beta) - sum(log(df$population)))
\end{lstlisting}

\subsection{グリッド探索を用いた推定}

次のグラフのように,尤度が最大になる時の$\alpha$は$1.3$となり,概ね解析会を用いた推定値と一致する.

\begin{figure}[H]
  \centering
  \includegraphics[width = \hsize]{\$HOME/code_box/learn/R/problemset2_2125178/02_code/output/grid_search.png}
\end{figure}

並び替えた完成系の $simulate\_mle$関数は以下のようになった.


\begin{lstlisting}[caption=解析解を用いた推定,label=c]
# (c)
simulate_mle <- function(df, beta, sample_size) {
  a <- "((alpha + 1)*sum(log(df$population))))"
  b <- "sample_size*alpha*log(beta)"
  c <- "(sample_size*log(alpha)"
  d <- "-"
  e <- "+"

  # 以下のコードを正しく並べ替えること
  df_likelihood <- dplyr::tibble(
    alpha = seq(0.1, 2, by = 0.01),
    l = eval(parse(text = paste0(c, e, b, d, a)))
  ) |>
    dplyr::mutate(
      alpha_hat = dplyr::if_else(max(l) == l, 1, 0)
    )


  ggplot() +
    geom_point(data = df_likelihood |> dplyr::filter(alpha_hat == 1), mapping = aes(x = alpha, y = l), color = "red") +
    geom_point(data = df_likelihood, mapping = aes(x = alpha, y = l), alpha = 0.2) +
    geom_text(data = df_likelihood |> dplyr::filter(alpha_hat == 1), mapping = aes(x = alpha, y = l, label = alpha), hjust = 0, vjust = -1) +
    theme_bw()
}
\end{lstlisting}

\subsection{勾配法を用いた推定}

\subsection{パレート分布の仮定の検証(Adv.)}

\end{document}


